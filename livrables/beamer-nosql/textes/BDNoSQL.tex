% //////////////////////////////////////// %
% /// Au delà du SQL : les BDs NoSQL ///// %
\section{Au delà du SQL : les BDs NoSQL}

    %% Pourquoi changer du relationnel ?
    \subsection{Pourquoi changer du relationnel ?}
    \begin{frame}
        \frametitle{Problèmes liés au relationnel}

        \begin{itemize}
            \item Service réparti sur plusieurs continents
            \item Accès concurrent de plusieurs centaines de milliers de personnes
            \item Stockage de petites variations dans les schémas
            \item Temps de réponse
            \item Traitement de structures de données complexes en base de données
        \end{itemize}
    \end{frame}

    \begin{frame}
        \frametitle{Pourquoi changer du relationnel ?}

        Besoin d'une alternative vers les années 2004 avec l'arrivée du \textit{Big Data}.
        \begin{itemize}
            \item Des volumes de données important (plusieurs gigas, téraoctets, pétaoctets) ;
            \item Un nombre de transactions très important, une forte demande de disponibilité et de temps de réponse ;
            \item Des bases de données réparties sur plusieurs centres de données ou continents ;
            \item Préférence pour l'ajout de petites machines plutôt qu'une configuration poussée des BDs (concept de \textbf{scalabilité horizontale}).
        \end{itemize}
    \end{frame}

    \begin{frame}
        \frametitle{Pourquoi ne pas changer du relationnel ?}

        \begin{itemize}
            \item Technologie éprouvée ;
            \item Technologie \textbf{très} éprouvée ;
        \end{itemize}

        \begin{itemize}
            \item Gestion jusqu'à 2To de RAM sur une seule machine ;
            \item Réplication de PostgreSQL sur plusieurs continents via master read-write et slaves read-only ;
            \item Ajout du type JSON pour relaxer les schémas ;
            \item Secteurs où la fiabilité des données est plus importante que le reste.
        \end{itemize}
    \end{frame}

    %% Relâchement des contraintes de transactions
    \subsection{Relâchement des contraintes de transactions}
    \begin{frame}
        \frametitle{Relâchement des contraintes de transactions}

        Les BDs relationnelles et NoSQL font des arbitrages différents dans le théorème CAP :
        \begin{itemize}
            \item \textit{Consistency} : chaque lecture reçoit l'écriture la plus récente ou une erreur ;
            \item \textit{Availability} : chaque requête reçoit une réponse, sans garantie qu'elle contienne l'écriture la plus récente ;
            \item \textit{Partition tolerance} : le système continue de fonctionner malgré la perte ou le retard de messages entre les nœuds.
        \end{itemize}
    \end{frame}

    %% À quoi ressemble une BD NoSQL
    \subsection{À quoi ressemble une BD NoSQL}
    \begin{frame}
        \frametitle{À quoi ressemble une BD NoSQL}

        \begin{itemize}
            \item Un SGBD qui n'est pas structuré en tables et dont l'élément de base n'est pas un tuple mais dépend du type de BD NoSQL ;
            \item Un langage de requête non uniformisé, propre à chaque BD ;
            \item Une dénormalisation des données où certains enregistrements sont en partie ou entièrement dupliqués ;
            \item Type de base de données NoSQL à choisir en fonction de l'usage souhaité ;
            \item Types de base de données NoSQL existants : clé-valeur (Redis), colonnes (Cassandra), recherche (ElasticSearch), documents (MongoDB), graphe (Neo4j), évènements (Event Store)\dots
        \end{itemize}
    \end{frame}
