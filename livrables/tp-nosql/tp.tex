\documentclass[a4paper,10pt]{article}
\usepackage{fullpage}
\usepackage{vmargin}
\usepackage[utf8]{inputenc}
\usepackage[french]{babel}
\usepackage{amsmath}
\usepackage{pifont}
\usepackage{verbatim}
\usepackage{graphicx}
\usepackage{tikz}
\usepackage{csquotes}
\usetikzlibrary{positioning,shadows,arrows,fit,automata,fadings,patterns,shapes}

% Couleurs prédéfinies
\colorlet{vertP}{green!10!white}
\colorlet{vertF}{green!90!black}
\colorlet{vertM}{green!40!white}
\definecolor{gazon}{RGB}{58,157,35}
\colorlet{bleuP}{blue!10!white}
\colorlet{bleuF}{blue!90!black}
\colorlet{bleuM}{blue!40!white}
\colorlet{rougeP}{red!10!white}
\colorlet{rougeM}{red!40!white}
\colorlet{rouge}{red!80!black}
\colorlet{rougeF}{red!90!black}
\colorlet{bleu}{blue!80!black}
\colorlet{vert}{green!80!black}
\colorlet{jaune}{yellow!80!black}
\colorlet{jauneF}{yellow!90!black}
\colorlet{jauneM}{yellow!40!white}
\colorlet{jauneP}{yellow!10!white}
\colorlet{grisP}{gray!20!white}
\definecolor{grisclair}{RGB}{206,206,206}
\definecolor{orangebis}{RGB}{222, 41, 22}
\definecolor{roux}{RGB}{173,79,9}
\definecolor{lilas}{RGB}{182,102,210}
\colorlet{lilasP}{lilas!20!white}
\colorlet{lilasM}{lilas!40!white}
\colorlet{jauneJoli}{yellow!50!white}
\definecolor{bordeaux}{RGB}{109,7,26}
\definecolor{magenta}{RGB}{255,0,255}

\usepackage[tikz]{bclogo}
\renewcommand\styleSousTitre[1]{\scriptsize\textsc{#1}}
\usepackage{placeins}
\usepackage{url}

\title{TP bases de données NoSQL}
\date{}
\author{Antoine Augusti \& Thibaud Dauce}

\begin{document}

\maketitle
\sloppy
\begin{bclogo}[logo = \bclampe, arrondi = 0.1, ombre = true, epOmbre = 0.2, couleurOmbre = black!30, couleurBord=bleuF, barre=none]{Objectifs}
\begin{itemize}
 \item Découvrir une base de données clé-valeur avec Redis
\end{itemize}

\end{bclogo}


\begin{bclogo}[logo = \bccrayon, noborder=true,barre=none]{}
  Bloc simple
\end{bclogo}

\section{Redis}
  \begin{bclogo}[logo = \bctakecare, arrondi = 0.1, ombre = true , epOmbre = 0.1, couleurOmbre = black!30,  barre =none, couleurBarre=bleuF]{Commandes et documentation}
    \begin{itemize}
    \item[$\bullet$] Les commandes Redis à notre disposition sont les suivantes, à taper dans un terminal :
    \begin{verbatim}
      /opt/redis-4.1/src/redis-server
      /opt/redis-4.1/src/redis-cli
      /opt/redis-4.1/src/redis-benchmark
    \end{verbatim}
    \item[] Ces commandes permettent de : lancer un serveur Redis en local, lancer un client Redis (qui se connecte par défaut en local) et de lancer un test de performance sur un serveur Redis.
    \item[$\bullet$] Dans un client Redis, tapez la commande \texttt{HELP nom\_commande} pour connaître la documentation de la commande \textit{nom\_commande}.
    \item[$\bullet$] Vous pouvez trouver toutes les commandes Redis disponibles à l'adresse suivante : \url{https://redis.io/commands}.

   \end{itemize}
  \end{bclogo}

  Redis est une base de données de type clé-valeur, un des types de bases de données NoSQL. La fonction d'une base de données clé-valeur est de pouvoir stocker des données (appelée \textit{valeur}) dans une \textit{clé}. On peut retrouver ensuite la valeur stockée précédemment grâce à la clé, seulement si l'on connaît le nom de la clé qui contient la donnée. Il n'y a pas de façon directe de chercher une clé.\\

  On peut voir ceci comme un dictionnaire géant, mais où le stockage est persistant. Ceci signifie que si l'on redémarre un serveur Redis, les données seront toujours présentes.
  \subsection{Découverte de Redis}
    \begin{enumerate}
      \item Lancez un serveur Redis dans un terminal. Ouvrez un autre terminal et lancez un client Redis, puis exécutez la commande \texttt{PING} (\url{https://redis.io/commands/ping})
      \item Redis supporte plusieurs types d'objets : chaînes de caractères, entiers, listes, ensembles, ensemble ordonné, dictionnaire etc. Familiarisez-vous avec ces différents types et les commandes Redis à l'aide du tutoriel à l'adresse \url{https://try.redis.io}
      \item Redis est incroyablement rapide, sans configuration avancée, même avec une machine basique. Nous allons lancer un test de performance localement pour observer cela.\\

      Vérifiez que votre serveur Redis local tourne toujours (sinon lancez-le à nouveau). Dans une autre session de terminal, lancez la commande \texttt{redis-benchmark}. Si certaines commandes vous sont inconnues, référez-vous à la documentation générale des commandes Redis. Prêtez attention à la complexité algorithmique des opérations, indiquée sur la documentation.
    \end{enumerate}

\end{document}
