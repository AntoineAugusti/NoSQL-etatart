	%% Conclusion
	\subsection{Bonnes pratiques}
	\begin{frame}
		\frametitle{Les bonnes pratiques du NoSQL}

		\begin{itemize}
			\item Bien connaître ses données ;
			\item Ne pas avoir peur de la redondance ;
			\item Ne pas trop redonder ;
			\item Et surtout, bien quantifier ses besoins.
		\end{itemize}


		\begin{alertblock}{RTFM}
			La majorité des bases de données NoSQL connues sur le marché possèdent une très bonne documentation, souvent faite à destination de personnes venant du monde du relationnel.
		\end{alertblock}
	\end{frame}
	\subsection{Conclusion}
	\begin{frame}
		\frametitle{Conclusion : NoSQL ou relationnel}

		\begin{alertblock}{Les deux mon capitaine !}
			Les bases de données NoSQL ont été inventées afin de résoudre des problèmes insolubles par les bases de données relationnelle et non pas pour les remplacer.
		\end{alertblock}

		\vspace{20px}
		
		\begin{tabular}{|l|l|}
			\hline
			\textbf{NoSQL} & \textbf{Relationnel} \\ \hline\hline
			stockage de masse & stockage fiable \\ \hline
			données diverses & données formatées \\ \hline
			scalabilité horizontale & scalabilité verticale \\ \hline
		\end{tabular}

	\end{frame}

	\begin{frame}
		\frametitle{Conclusion : choix du type de BD NoSQL}

		\begin{tabular}{|p{0.20\textwidth}|p{0.40\textwidth}|p{0.30\textwidth}|}
			\hline
			& Modélisation & Cas d'utilisation \\\hline
			Clé / valeur
			& Modélisation simple, permettant d'indexer des informations diverses via une clé
			& Mise en cache  \\\hline
			Documents
			& Modélisation souple permettant de stocker des documents au format JSON dans des collections
			& Stockage de masse \\\hline
			Graphes
			& Modélisation optimisée pour les problèmes de graphes
			& Stockage provisoire pour traiter les données \\\hline
		\end{tabular}
	\end{frame}