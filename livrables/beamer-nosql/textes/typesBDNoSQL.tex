% ///////////////////////////////////////////// %
% /// Les types de bases de données NoSQL ///// %
\section{Les types de bases de données NoSQL}

	%% Bases de données clé-valeur
	\subsection{Bases de données clé-valeur}
	\begin{frame}
		\frametitle{Bases de données clé-valeur}

		\begin{itemize}
			\item Modélisation la plus simple. À une clé, on associe une valeur. La valeur peut être de n'importe quel type (chaîne de caractères, entier, structure, objet sérialisé\dots) ;
			\item Opérations basiques : création d'une paire clé-valeur, suppression, accès à une valeur à l'aide de la clé, incrémentation et décrémentation d'une valeur ;
			\item On peut définir la durée de vie d'une clé ou adopter une politique \textit{least recently used} ;
			\item Cas d'utilisation : cache d'une autre BD, comptage d'éléments, gestion de files d'attente, opérations ensemblistes\dots
			\item Principaux acteurs : Redis, Memcached, Riak.
		\end{itemize}

	\end{frame}

	\begin{frame}
		\frametitle{Exemple de BD clé-valeur}

		Des exemples de paires clé-valeur avec des types différents.
		\vspace{15px}

		\begin{tabular}{|l|l|}
			\hline
			\textbf{Clé} & \textbf{Valeur} \\ \hline\hline
			pays.id-42 & \{"id":42,"name":"Chad"\} \\ \hline
			statistiques.nombre-visiteurs & 1337 \\ \hline
			configuration.periode-gratuite & false \\ \hline
			articles.categories-sport.latest & [22, 45, 67, 200, 87] \\ \hline
		\end{tabular}

	\end{frame}

	\begin{frame}
		\frametitle{Exemple de mise en cache}

		\begin{listing}[H]
			\inputminted[fontsize=\tiny, linenos=true]{php}{code/cacheRepository.php}
			\caption{Mise en cache d'un pays par son ID.}
		\end{listing}

	\end{frame}