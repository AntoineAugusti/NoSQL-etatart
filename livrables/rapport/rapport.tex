\documentclass[12pt,titlepage,a4paper]{report}

% Texte
\usepackage[utf8]{inputenc}
\usepackage[T1]{fontenc}
\usepackage[french]{babel}
\usepackage[babel=true]{csquotes}
\usepackage{lmodern}
\usepackage{minted}
\usemintedstyle{trac}

% Numéroter les chapitres a partir de chaque début de partie
\makeatletter\@addtoreset{chapter}{part}\makeatother

% Mise en page
\usepackage{url}
\usepackage[top=2.1cm,bottom=2cm,left=1cm,right=1cm]{geometry}
\usepackage{hyperref}
\hypersetup{
    colorlinks=false,
    pdfborder={0 0 0},
}
\usepackage{multirow}

% TOC
\usepackage[french]{minitoc}
\setcounter{tocdepth}{2}
\setcounter{minitocdepth}{2}
\setlength{\mtcindent}{0pt}

% Images
\usepackage{float}
\usepackage{wrapfig}
\usepackage{graphicx}
% Pour inclure des pages PDF
\usepackage[final]{pdfpages}

% Couverture
\usepackage{templateINSA}
\initINSA

% Citations
\usepackage{epigraph}
\setlength\epigraphwidth{12cm}
\setlength\epigraphrule{0pt}

\usepackage{etoolbox}

\makeatletter
\patchcmd{\epigraph}{\@epitext{#1}}{\itshape\@epitext{#1}}{}{}
\makeatother

\usepackage[nottoc, notlof, notlot]{tocbibind}

\title{Les bases de données NoSQL}
\author{Antoine \bsc{Augusti}\\ Thibaud \bsc{Dauce}}

\renewcommand\soustitre{État de l'art}
\renewcommand\infoBig{PAO}
\renewcommand\infoSmall{ASI4 2014-2015}

\def\changemargin#1#2{\list{}{\rightmargin#2\leftmargin#1}\item[]}
\let\endchangemargin=\endlist

%% -- Document
\begin{document}
	\titleINSA{15}{images/fond.png}{0}{0}{300}{\href{http://upload.wikimedia.org/wikipedia/commons/5/59/Wikimedia_Foundation_Servers-8055_14.jpg}{\textcolor{white}{Licence CC - Wikimedia}}}
	\dominitoc
	\tableofcontents

	\chapter{Pourquoi le NoSQL}
	\minitoc
	
		\section{L'émergence du Big Data}
		\epigraph{``Big Data is like teenage sex: everyone talks about it, nobody really knows how to do it, everyone thinks everyone else is doing, so everyone claims they are doing it."}{\textup{\symbol{64}achille\_z}, Twitter}

Aux débuts des années 2000, avec la généralisation des interconnexions de réseaux, l'augmentation de la bande passante sur Internet et la diminution des coûts des machines moyennement puissantes, de nouvelles possibilités ont vu le jour dans le domaine de l'informatique distribuée\footnote{L’architecture d'un environnement informatique ou d'un réseau est dite distribuée quand toutes les ressources ne se trouvent pas au même endroit ou sur la même machine.\cite{Wikipedia_architecture_distribuee}} et de la virtualisation par exemple.\\

Le volume de données manipulées par certaines entreprises, notamment celles en rapport avec Internet a augmenté considérablement. Il était alors utopique de songer à stocker ses données sur une seule machine, celle-ci ne pouvant être suffisamment puissante pour pouvoir gérer une telle quantité d'information. L'informatisation croissante des traitements implique une multiplication exponentielle du volume de données qui se compte maintenant en pétaoctets (1 000 téraoctets)\footnote{En juin 2012 Facebook annonçait que son installation de Hadoop atteignait le volume physique de 100 pétaoctets.\cite{facebook_hadoop}}.\\

Les Anglo-Saxons ont nommé ce phénomène le \textit{Big Data}. La gestion et le traitement de ces volumes de données sont considérés comme un nouveau défi de l'informatique. Les moteurs de bases de données relationnels traditionnels, hautement transactionnels semblent dépassés par ces nouvelles contraintes.

		\section{Scalabilité horizontale et MapReduce}
		Depuis que les données sont réparties sur plusieurs centaines de machines, il a été nécessaire de trouver des moyens de répartir les calculs dans le cluster\footnote{En calcul distribué, le cluster est un système informatique composé d'unités de calcul (micro-processeurs, cœurs, unités centrales) autonomes qui sont reliées entre elles à l'aide d'un réseau de communication.\cite{Wikipedia_cluster}} et d'agréger les résultats de ces calculs pour produire une réponse finale. MapReduce est un principe et un algorithme utilisé pour ces besoins.

\subsection{Scalabilité horizontale}
	La scalabilité verticable désigne la possibilité d'augmenter les performances d'un serveur (ajout de processeurs, RAM, disques\dots) tandis que la scalabilité horizontale désigne la possibilité d'ajouter des serveurs d'un type donné.\cite{Wikipedia_scalabilite}. Ces dernières années, la diminution du coût du matériel a été une formidable opportunité pour cette dernière qui devient alors beaucoup plus intéressante.\\

	Optimiser les performances d'un SGBD\footnote{SGBD : Système de Gestion de Bases de Données} sur une seule machine demande beaucoup d'énergie et de compétences pour aboutir à des résultats fragiles et qui ne peuvent supporter une multiplication soudaine de la demande. En revanche, s'assurer que le modèle de traitement de données est bien distribué de la façon la plus élégante possible sur des machines séparées, qui peuvent être multipliées à l'infini, permet de répondre à des augmentations éclairs de la demande par l'achat et l'installation rapide de nouvelles machines (puissantes ou non) et en s'assurant que toute défaillance d'une machine ne se traduise pas par une perte de données.\\

	Il faut donc vérifier que le modèle déployé est capable de distribuer au mieux les données et le travail, même sur plusieurs milliers de nœuds, en offrant un système de réplication suffisant pour éliminer statistiquement le risque de perte de données.

\subsection{Pourquoi MapReduce ?}
	Comment traiter des volumes gigantesques de données, réparti sur plusieurs machines dans plusieurs centres de données pour en tirer des résultats de calculs, d'agrégats, de résumés\dots ? MapReduce a été défini par Google pour répondre à ces besoins.

\subsection{Principe de MapReduce}
	MapReduce a été défini en 2004 dans un brevet de Google.\cite{google_mapreduce}. Le principe est simple : pour distribuer un traitement, Google a imaginé une opération en deux étapes :
	\begin{enumerate}
		\item L'attribution des opérations à effectuer sur chaque machine (étape \texttt{Map}) ;
		\item Rassemblement des résultats après l'étape de traitement (étape \texttt{Reduce}).
	\end{enumerate}

	En soi, le raisonnement n'est pas révolutionnaire et n'a pas été inventé par Google. Les opérations de \texttt{map} et de \texttt{reduce} ont été inspirées par les primitives du même nom en Lisp. Le brevet de Google répond néanmoins aux problématiques de MapReduce dans un environnement distribué. Que faire en cas de défaillance d'une unité de traitement ? Comment s'assurer d'une bonne distribution du travail ? Comment synchroniser les résultats des traitements ?

\subsection{Hadoop, une implémentation de MapReduce}
	L'implémentation la plus connue de MapReduce est le framework Java libre Hadoop. Hadoop utilise totalement le principe de la scalabilité horizontale puisque Hadoop a été conçu pour fonctionner avec plusieurs milliers de machines au sein d'un même cluster.\\

	Hadoop a été créé par Doug Cutting qui s'est inspiré du GoogleFS\footnote{GoogleFS : \textit{Google File System}. Google File System (GFS) est un système de fichiers distribué propriétaire. Il est développé par Google pour leurs propres applications.}. Hadoop utilise le HDFS\footnote{HDFS : \textit{Hadoop Distributed File System.}}, système de fichiers distribué extensible. Il a été conçu pour stocker de très gros volumes de données sur un grand nombre de machines. Aujourd'hui le plus gros cluster Hadoop est exploité par Yahoo : celui-ci compte 10 000 machines.\cite{yahoo_hadoop} Facebook utilise également Hadoop et stocke plus de 100 000 téraoctets sur son cluster.\cite{facebook_hadoop}


		\section{Une approche non relationnelle}
			Le modèle de données relationnel et celui associé aux bases de données NoSQL sont très différents. Le modèle relationnel sépare les données au sein de plusieurs tables qui contiennent des lignes et des colonnes. Les tables se référencent entre elles à l'aide de clés étrangères stockées sous la forme d'attributs. Quand on recherche des données, l'information a éventuellement besoin d'être collectée depuis plusieurs tables (parfois plusieurs dizaines ou centaines pour des modèles de données d'entreprise) et rassemblée avant d'être envoyée à l'application. C'est ce que l'on appelle les jointures. De manière similaire, lors d'une écriture, l'écriture doit souvent être effectuée sur plusieurs tables.\\

% \subsection*{La modélisation NoSQL}
% 	Les bases de données ont une modélisation des données très différente du modèle relationnel. Ces différences sont tellement importantes que nous les étudierons dans un chapitre dédié. Par exemple, dans une base de données orientée document, les données sont agrégées dans un document en utilisant le format JSON.\\

% 	Un document JSON peut être imaginé comme un objet prêt à être utilisé dans l'application finale. Un document JSON peut par exemple prendre toutes les données d'une ligne dans 10 tables d'une base de données relationnelle et les agréger dans un seul document. L’agrégation de cette information conduit à la duplication de l'information mais depuis que les coûts de stockage sont devenus faibles, la flexibilité de la modélisation des données, la facilité d'utilisation des documents et les améliorations en lecture / écriture facilitent beaucoup le stockage de données pour des applications web.

% 	\begin{listing}[H]
% 		\inputminted{json}{code/commentaire.json}
% 		\caption{Un commentaire sur une citation au format JSON. Exemple tiré de l'API de Teen Quotes.\cite{teenquotes_api_commentaire}}
% 	\end{listing}
% 	Dans ce document on voit que les données de l'utilisateur auteur du commentaire et de la citation sur laquelle ce commentaire a été posté ont été enchâssées (principe de l'\textit{embedding}).\\

% 	À titre de comparaison, la requête SQL équivalente pour obtenir le résultat précédent serait la suivante.
% 	\begin{listing}[H]
% 		\inputminted{sql}{code/commentaireSQL.sql}
% 		\caption{La requête SQL équivalente au résultat précédent.}
% 	\end{listing}
% 	Ce résultat, tout à fait ordinaire, requiert déjà 4 jointures pour une base de données relationnelles. Il est ensuite nécessaire d'effectuer des modifications par rapport au retour brut de la base de données pour former des objets imbriqués qui seront exploités par l'application.

Le modèle relationnel repose sur une stabilité du schéma des données et donc sur une modélisation poussée, idéalement complète, effectuée au préalable. Il est malheureusement difficile de respecter cette contrainte à la perfection et il est quasiment inimaginable de ne pas faire évoluer son schéma au fil des années, suite à la première mise en production. Changer le schéma d'une base de données relationnelle est coûteux afin de garantir une cohérence avec les applications déjà existantes et est, de ce fait, très souvent évité. Ceci est totalement opposé au comportement désiré à l'ère du \textit{Big Data} où les développeurs doivent constamment et rapidement incorporer de nouveaux types de données pour enrichir les applications. Ainsi, l'ajout d'une nouvelle source d'information peut se faire plus rapidement car on ne doit pas avoir exactement la même structure de données que les autres sources d'information utilisées pour remplir la base de données. Cette souplesse que l'on s'autorise possède bien évidemment des inconvénients.\\

Les bases de données NoSQL cherchent à assouplir la contrainte d'un schéma figé en proposant une approche dite \enquote{sans schéma} ou à schéma \enquote{relaxé}, ce qui veut dire que le moteur n'effectue aucune vérification ou contrainte de schéma. C'est alors au développeur qui utilise ces systèmes de décider comment il organise les données, et ceci dans son code client.\\

Dans les faits il est quand même difficile de s'offrir une totale liberté. Il est fortement recommandé de conserver une structure homogène de données dans la même collection ou la même base pour des raisons de logique et notamment d'indexation. Tout mélanger n'aurait aucun sens. Comment effectuer une recherche avec des critères précis sans laisser de côté une partie des enregistrements mal structurés ?\\

Pour le modèle relationnel, il est important que les données soient exactes en tout temps. Elles sont donc structurées, contraintes, protégées et cohérentes. En stockant vos données dans une base NoSQL, vous leur accordez un écrin différent, plus permissif. Cette permissivité permet la densification des accès concurrents. Les données ainsi accédées sont susceptibles d'être inexactes, il faut donc être vigilant sur la sensibilité et les responsabilités qu'ont ces données. Dans l'e-commerce par exemple, là où une base de données relationnelle devra attendre la fin d'une transaction pour afficher la disponibilité du produit à un autre client, une base de données NoSQL permettra l'affichage concurrent au risque de se retrouver pendant un court instant avec des stocks affichés ne reflétant pas la réalité. Cet écart à la réalité peut être plus dangereux dans le cadre d'un lancement de fusée par exemple. C'est le choix entre la cohérence à tout instant ou la relaxation des propriétés ACID\footnote{Propriétés ACID : propriétés contraignant les accès concurrents. Voir sous-section \ref{subsec:proprietesACID}.} permettant une plus grande flexibilité. 
\epigraph{``Vous êtes en quelque sorte comme un parent cool qui laisse la chambre de ses enfants dans l'ordre qu'ils souhaitent. Dans le monde de la base de données, les enfants sont les développeurs et les données représentent la chambre."}{\textup{Rudi Bruchez}, Les bases de données NoSQL - Comprendre et mettre en œuvre \cite{BD_NoSQL}}


	\chapter{NoSQL et SQL : quelles différences ?}
		
		\section{Les principes du relationnel en regard du NoSQL}
			\subsection{La normalisation}
			\subsection{Le défaut d'impédance}
			\subsection{Les NULL}
		
		\section{Transactions et cohérence des données}
			\subsection{Isolation de la transaction}
			\subsection{La transaction distribuée}
			\subsection{Le théorème CAP}

	\chapter{Les différents schémas de bases de données dans les bases NoSQL}
		\section{Les bases de données clé / valeur}
		\section{Les bases de données orientées documents}
		\section{Les bases de données orientées graphes}
	
	%% -- Bibliographie
	\bibliographystyle{plain}
	\bibliography{bib}
\end{document}