\subsection{Les transactions et les propriétés ACID}
\label{subsec:proprietesACID}
	Une transaction est \enquote{un ensemble d'opérations élémentaires dont l'exécution provoque le passage d'un état cohérent de la base de données à un autre état cohérent.}\cite{cours_bdd_insa} Les transactions visent à préserver l'intégrité des données dans un environnement multi-usagers, non fiable, où des pannes peuvent se produire.\\

	Les bases de données relationnelles cherchent à maintenir une cohérence forte de la base de données en respectant les propriétés dites ACID, définies en 1983 par Andreas Reuter et Theo Härder. Ces propriétés sont les suivantes :
	\vspace{10px}
	\begin{itemize}
		\item \textbf{Atomicité.} Une transaction doit s'effectuer en intégralité ou pas du tout. Si une partie d'une transaction ne peut être faite, il faut effacer toute trace de la transaction et remettre les données dans l'état où elles étaient avant la transaction.
		\item \textbf{Cohérence.} Chaque transaction doit amener le système d'un état valide à un autre état valide. La validité de l'état est, entre autre, assurée par les contraintes d'intégrité.
		\item \textbf{Isolation.} Les transactions ne doivent pas voir les résultats d'autres transactions en cours. Elles s'exécutent comme si elles étaient seules sur le système.
		\item \textbf{Durabilité.} Aucun autre événement, hormis une transaction de compensation ne peut supprimer les effets d'une transaction terminée et correctement exécutée.
	\end{itemize}
	\vspace{20px}
	Le strict respect des propriétés ACID par une base de donnée dégrade ses performances. La garantie des propriétés ACID dans un système distribué de transactions à travers une base de donnée, elle-même distribuée, présente des complications additionnelles. Ces problèmes seront détaillées dans la partie \ref{subsec:theoremeCAP} où nous discuterons du théorème CAP.

\subsection{Isolation de la transaction}

	Comme vu dans le paragraphe précédent, une transaction isole les données accédées afin d'éviter une corruption des données due à une modification simultanée par plusieurs utilisateurs concurrents. Cette propriété est gérée, dans le cas de lectures concurrentes, de différentes manières en fonction des bases de données utilisées. Dans certains cas, elle provoque des temps d'attente, dans d'autres, des complications en obligeant le moteur à garder en mémoire des états intermédiaires. En revanche, deux écritures simultanées sont interdites. Le verrouillage peut se situer à plusieurs niveaux : base, table ou, dans le meilleur des cas, ligne.\\

	Il existe trois problématiques lors d'accès concurrents à une base de données :
	\vspace{10px}
	\begin{itemize}
		\item \textbf{Lecture sale} : \enquote{lecture par une requête de données en cours de modification par une autre}\cite{wiktionnaireLectureSale} ;
		\item \textbf{Lecture non reproductible} : \enquote{relecture par une requête de données qui ont été modifiées par une autre depuis la première lecture}\cite{wiktionnaireLectureNonReproductible} ;
		\item \textbf{Lecture fantôme} : \enquote{cas de différences entre deux exécutions d'une requête, car les données ont été modifiées par une autre entre temps}\cite{wiktionnaireLectureFantome}.
	\end{itemize}
	\vspace{10px}

	 Ces problèmes étant courants, la norme SQL a défini 4 niveaux d'isolation afin de les résoudre le plus efficacement possible :
	\vspace{10px}
	\begin{itemize}
		\item \textbf{Uncommited Read} (en français, \textit{Lecture de données non validées}) : permet tout ;
		\item \textbf{Commited Read} (en français, \textit{Lecture de données validées}) : empêche uniquement les lectures sales ;
		\item \textbf{Repeatable Read} (en français, \textit{Lecture répétée}) : laisse uniquement passer les problèmes de lectures fantômes ;
		\item \textbf{Serializable} (en français, \textit{Sérialisable}) : résout tous les problèmes.
	\end{itemize}
	\vspace{20px}
	 Les bases de données implémentent tout ou partie de ces niveaux (PostgreSQL implémente les niveaux \textit{Commited Read} et \textit{Serializable}\cite{isolationTransactionPostgre} alors que SQL Server, par exemple, les implémentent tous\cite{isolationTransactionSQLServer}). Il est donc nécessaire de se renseigner sur les caractéristiques des différentes bases de données au niveau de l'isolation des transactions avant de faire son choix.\\

	Dans le monde du NoSQL, les bases de données ne tiennent pas compte de la cohérence transactionnelle pour des raisons de simplicité et de performances afin de gérer de nombreux utilisateurs. Il faut préciser que même dans le monde relationnel, certains moteurs ne gèrent pas les transactions. C'est par exemple le cas du moteur de stockage MyISAM de la base de données MySQL. Dans la pratique, les lectures sales n'engendrent que très rarement des problèmes, nous avons donc ici une vision optimiste des transactions.
