% //////////////////////////////////////// %
% /// Au delà du SQL : les BDs NoSQL ///// %
\section{Au delà du SQL : les BDs NoSQL}

	%% Pourquoi changer du relationnel ?
	\subsection{Pourquoi changer du relationnel ?}
	\begin{frame}
		\frametitle{Pourquoi changer du relationnel ?}

		Besoin d'une alternative vers les années 2004 avec l'arrivée du \textit{Big Data}.
		\begin{itemize}
			\item Des volumes de données important (plusieurs gigas, voire téraoctets) ;
			\item Un nombre de transactions très important, une forte demande de disponibilité et de temps de réponse ;
			\item Des bases de données réparties sur plusieurs centres de données ou continents ;
			\item Préférence pour l'ajout de petites machines plutôt qu'une configuration poussée des BDs.
		\end{itemize}

	\end{frame}

	%% Relâchement des contraintes de transactions
	\subsection{Relâchement des contraintes de transactions}
	\begin{frame}
		\frametitle{Relâchement des contraintes de transactions}

		Impossible de garantir les propriétés ACID des BDs relationnelles avec les nouvelles contraintes.\\
		\vspace{10px}
		De nouvelles propriétés \textbf{BASE}, suite au théorème CAP :
		\begin{itemize}
			\item \textit{Basic Availability} : système disponible dans son ensemble bien que certaines machines soient indisponibles ;
			\item \textit{Soft state} : l'état du système distribué peut changer, même sans nouvelles transactions ;
			\item \textit{Eventual Consistency} : En l'absence de nouvelles transactions, le système sera cohérent au bout d'un temps.
		\end{itemize}

	\end{frame}

	%% À quoi ressemble une BD NoSQL
	\subsection{À quoi ressemble une BD NoSQL}
	\begin{frame}
		\frametitle{À quoi ressemble une BD NoSQL}

		\begin{itemize}
			\item Un SGBD qui n'est pas structuré en tables et dont l'élément de base n'est pas un tuple mais dépend du type de BD NoSQL ;
			\item Un langage de requête non uniformisé, propre à chaque BD. Souvent au format JSON avec une API REST ;
			\item Une dénormalisation des données où certains enregistrements sont en partie ou entièrement dupliqués ;
			\item Type de base de données NoSQL à choisir en fonction de l'usage souhaité.
		\end{itemize}

	\end{frame}