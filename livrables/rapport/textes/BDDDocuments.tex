Les bases de données orientées documents sont les bases de données NoSQL qui ressemblent le plus aux bases de données relationnelles. Chaque document peut être considéré comme un tuple dans une table d'une base de donnée relationnelle. Ces documents sont la plupart du temps stocké dans le format JSON, un format structuré et lisible par le moteur NoSQL. De nombreuses données peuvent être stockées dans un document : des textes, des chiffres mais aussi des tableaux ou encore d'autres documents. Contrairement aux tuples d'une base de données relationnelle classique, un document ne possède pas de structure prédéfinie, chaque document peut donc être différent et présenter des données variables.

\subsection{Le format JSON}

  Le format JSON pour \textit{JavaScript Objet Notation} est un format de représentation de données similaire au XML (\textit{Extensible Markup Language}). Il permet aussi bien de sérialiser l'état des classes que de structurer des données. Son format texte le rend reconnaissable et lisible sur toutes les plate-forme et tout système. Il est aussi très simple à traiter (génération et \textit{parsing}). Il est préféré au XML dans les bases de données NoSQL pour quelques raisons :
\begin{enumerate}
  \item Plus compact
  \item Moins verbeux
  \item Plus simple à lire et à manipuler
\end{enumerate}

\begin{listing}[H]
  \inputminted{json}{code/commentaire.json}
  \caption{Exemple de document JSON.}
\end{listing}

  Comme on le voit dans l'exemple, le JSON utilise des accolades et des crochets pour représenter l'information.
