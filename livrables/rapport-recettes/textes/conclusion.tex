Nous avons été très heureux de réaliser ces 2 PAO portant sur le NoSQL au cours de notre année d'ASI 4. Grâce à ces PAO, nous avons pu nous former et découvrir un nouveau type de bases de données qui ont permis d'approfondir notre connaissance des solutions de stockage des données.\\

Lors de l'utilisation de bases de données NoSQL, nous avons pu nous apercevoir qu'il était plus \enquote{compliqué} d'avoir accès à de la documentation, des clients optimisés pour les frameworks ou une communauté importante pour les différents types de bases de données NoSQL. Ceci s'explique par la relative nouveauté des bases de données NoSQL, mais aussi par la segmentation forte du marché selon les différents types (clé-valeur, documents, graphes\dots). Toutefois, nous n'avons pas rencontré de réels problèmes de mise en place et d'exploitation.\\

Nous avons confirmé que la majeure partie du travail était réalisée lors de la réflexion autour de l'utilisation de l'application (ces données seront-elles souvent lues ? Seront-elles souvent modifiées ?), qui permet ensuite de choisir une ou plusieurs bases de données NoSQL, puis de modéliser comment seront stockées les données. Nous avons trouvé ce travail plus complexe que pour les bases de données relationnelles car il n'existe aucun vérité concernant les hypothèses d'utilisation d'application et que les modélisations de bases de données NoSQL ne sont pas strictes.\\

Par ailleurs, il est important de rappeler que la modélisation avec des bases de données NoSQL que nous avons réalisé pour l'application de recettes est trop complexe pour l'utilisation actuelle de l'application. Le choix d'utiliser une base de données NoSQL doit être justifié par les circonstances et les besoins. La charge de l'application, le nombre d'utilisateurs, le niveau de formation des équipes, l'argent disponible sont autant de paramètres que nous n'avons pas pris en compte lorsque nous avons choisi d'utiliser des bases de données NoSQL.\\

Enfin, nous sommes persuadés de l'utilité des différents types de bases de données NoSQL, toutefois celles-ci doivent être utilisées dans des cas précis, que nous avons pu mettre en évidence au cours de nos PAO. Les bases de données NoSQL ne remplaceront vraisemblablement jamais les bases de données relationnelles, mais elles s'imposent comme des compléments indispensables aujourd'hui dans une architecture importante de système d'information.
