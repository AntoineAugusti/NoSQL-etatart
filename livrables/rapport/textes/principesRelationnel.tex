\subsection{La normalisation}

    Edgar Frank Codd est considéré comme l'inventeur du modèle relationnel. En 1970, il définit des règles simples et logiques afin de s'assurer que les modélisations des schémas relationnels sont correctes\cite{Wikipedia_Edgar_Frank_Codd}. Son livre définit des règles de normalisation sous la forme de 3 formes normales.
    \vspace{10px}
    \begin{itemize}
      \item \textbf{1\iere{} forme normale} : atomicité. Cette règle assure qu'une colonne d'une table dans un schéma relationnel ne contient qu'une valeur isolée afin de faciliter la recherche et la manipulation. Par exemple, il faut stocker \textit{Alice Dupont} dans deux champs (nom et prénom) afin de pouvoir rechercher par rapport au prénom, par rapport au nom ou par rapport aux deux.
      \item \textbf{2\ieme{} forme normale} : un attribut non clé d'une table doit dépendre de toutes les clés de la table et ne doit pas dépendre d'un sous-ensemble des clés de la table.
      \item \textbf{3\ieme{} forme normale} : un attribut non clé ne dépend pas d'un ou plusieurs attributs ne participant pas à la clé.
    \end{itemize}
    \vspace{20px}
    D'autres formes normales ont été défini par la suite, chacune d'entre elles précisant des bonnes pratiques supplémentaires dans la conception des bases de données relationnelles. Il est très difficile, voire impossible, de respecter ces recommandations de modélisation. Dans les bases de données NoSQL, nous allons relâcher ces formes normales afin de simplifier la structure de la base de données. Cette simplification vient à un prix : la perte de certaines optimisations comme par exemple la non-redondance des données stockées.

\subsection{Le défaut d'impédance}

  Le terme \enquote{défaut d'impédance} est tiré de l'électricité. L'impédance est, très simplement, une opposition faite par un circuit électrique au passage d'un courant. Mais qu'en est-il dans le cadre des bases de données relationnelles ? Le principal paradigme de programmation actuel en entreprise est la programmation orientée objet. Malheureusement, le langage SQL, bien que très puissant, pourrait être considéré comme un langage plat, textuel. Il est donc compliqué pour les développeurs de devoir sortir du modèle objet pour construire une requête SQL.\\

  Afin de palier à ce problème, des outils appelés ORM pour \textit{Object-Relational Mapping} ont été inventés. Ces outils permettent d'abstraire les requêtes SQL via une API\footnote{API : \textit{Application Programming Interface}} orientée objet et sont aujourd'hui considérés comme une bonne pratique indispensable dans tout projet informatique. Le défaut majeur de ce type d'approche est une forte résistance du langage SQL au changement de paradigme qui provoque des limitations conséquentes lors de l'utilisation de ces outils ce qui oblige les développeurs à déporter une partie de la gestion des données dans l'application client alors que ces traitements auraient pu être exécutés par le SGBDR plus efficacement.\\

  Les bases de données NoSQL ne répondent pas forcément à cette problématique mais de part leur conception simple, facilitent l'interrogation de la base. La majorité des bases de données NoSQL ne présentent pas d'intelligence poussée et se contentent de gérer l'accès aux données de manière efficace et optimisé. La méthode d'accès la plus courante est le format JSON\footnote{JSON : \textit{JavaScript Object Notation}} qui, bien que beaucoup moins puissant que le langage SQL, s'interface très facilement avec le paradigme orienté objet.


\subsection{Le marqueur NULL}
  Dans une base de données relationnelle, les marqueurs  NULL permettent d'indiquer l'absence de valeur. Selon la qualité de notre conception, ces marqueurs sont plus ou moins fréquents, mais comme pour le respect de toutes les formes normales, il est très compliqué de concevoir un schéma n'ayant pas besoin de cette petite astuce.\\

  Les marqueurs NULL sont très critiquées car ils ne sont pas équivalents à une chaîne de caractères vide ou à un 0. Ce concept présente deux défauts majeurs :
  \vspace{10px}
  \begin{itemize}
    \item \textbf{L'espace disque occupé} : le marqueur NULL, de part sa nature, occupe un espace plus ou moins important en fonction de la qualité de la base de données.
    \item \textbf{La gestion en SQL} : le concept de marqueur NULL oblige d'une part à la base de données de gérer un type de donnée supplémentaire et d'autre part aux développeurs de penser au traitement des valeurs NULL à chaque requête.
  \end{itemize}
  \vspace{20px}

  Dans une base de données NoSQL, le marqueur NULL n'est pas nécessaire. En effet, il est inutile d'indiquer l'absence de valeur car aucune valeur n'a été définie. Pour indiquer l'absence de valeur, il suffit de ne pas définir l'attribut concerné.
