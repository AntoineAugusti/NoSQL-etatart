Dans la première partie de notre Projet Approfondissement et d'Ouverture, nous avons étudié les différents types de bases de données NoSQL au travers d'un état de l'art. Ce document, disponible sur Github à l'adresse \url{https://github.com/AntoineAugusti/NoSQL-etatart/tree/master/livrables/rapport}, avait pour objectif de montrer les avantages et les inconvénients d'un maximum de bases de données différentes afin d'obtenir une vision globale de l'écosystème NoSQL aujourd'hui. Malgré les exemples utilisés pour illustrer les propos tenus, ce document garde une approche très théorique.\\

Suivant naturellement notre premier état de l'art, l'objectif de ce deuxième document porte sur la mise en pratique des bases de données NoSQL dans un projet réel. Pour cela, nous nous sommes appuyés sur le projet de base de données des étudiants ASI de 3ème année. Le sujet de cette année était la réalisation d'une application de gestion de recettes de cuisine ainsi que d'invitation à des événements culinaires. Les contraintes techniques des étudiants de 3ème année était la réalisation d'un client en C communiquant avec PostgreSQL, une base de données relationnelle. Nous avons trouvé intéressant de travailler sur le même sujet afin de pouvoir leur fournir une vision différente de la conception et de la modélisation d'une architecture de stockage de données en fonction du choix entre relationnel ou NoSQL.\\

Ce document est donc premièrement un rapport sur la modélisation et la conception de notre application, mais aussi un outil de travail pour les étudiants souhaitant approfondir leurs connaissances dans le domaine après notre intervention en cours de base de données.\\

Nous allons tout d'abord parler de Laravel, le framework web PHP retenu pour réaliser l'application, puis dans une deuxième partie, les choix effectués au niveau des bases de données NoSQL : les différents types de bases utilisés ainsi que les différentes modélisations possibles et retenues.