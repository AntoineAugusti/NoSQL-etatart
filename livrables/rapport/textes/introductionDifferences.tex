Comme nous l'avons vu dans la première partie, de l'émergence du \textit{Big Data} est né un éco-système complet allant du stockage des données à leur traitement et leur exploitation. Mais tous ces outils sont basé sur le concept du NoSQL qui permet de résoudre les problèmes de stockage, de traitement et de gestion distribué. Quels sont les éléments clés de ce concept et en quoi permettent-ils de changer la donne par rapport à une approche relationnelle ? 
