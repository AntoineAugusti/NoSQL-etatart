\documentclass[a4paper, 12pt, french]{article}
\usepackage[utf8]{inputenc}
\usepackage[french]{babel}
\usepackage[T1]{fontenc}
\usepackage[babel=true]{csquotes}
\usepackage{hyperref}
\usepackage{graphicx}
\usepackage{array}
\usepackage{fancyhdr}
\usepackage{fullpage}
\usepackage{float}

\title{Objectifs des PAO NoSQL}
\author{Antoine Augusti\\ Thibaud Dauce}
\date{\today}

%% -- Document
\begin{document}
	\maketitle

	\section{Objectifs des PAO}
	Comme discuté, l'idée est de partir sur deux PAO\footnote{PAO : Projets d'Approndissement et d'Ouverture} lors de notre année d'ASI 4 à partir de septembre 2014. Le premier PAO serait dédié à un état de l'art des bases de données NoSQL tandis que le deuxième PAO serait une mise en œuvre pratique d'une application utilisant des bases de données NoSQL.

	\section{État de l'art des bases de données NoSQL}
	Ce premier PAO se déroulerait durant le premier semestre d'ASI 4, de septembre à décembre 2014. Il aurait pour objectif de présenter les différents types de bases de données NoSQL existant actuellement :
	\vspace{10px}
	\begin{itemize}
		\item Base de données orientée objet
		\item Base de données orientée colonnes
		\item Base de données orientée documents
		\item Base de données orientée graphe
	\end{itemize}
	\vspace{10px}

	Nous aurons le souci de présenter les grands acteurs industriels et comment les données sont modélisées par rapport aux bases de données relationnelles pour chaque type de base de données.\\

	Le document que nous produirons pourra compléter le cours de base de données enseigné en ASI 3. Il pourra également servir de support pour le département ASI si des étudiants doivent mettre en place des bases de données NoSQL.

\end{document}