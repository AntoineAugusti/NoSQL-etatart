\subsection{Modélisation}
	Les bases de données clé-valeur sont les bases de données qui possèdent la représentation la plus simple car elles ne contiennent que des paires clé-valeur. Les opérations disponibles sur de telles bases de données sont très limitées :
	\begin{itemize}
		\item Création d'une paire clé-valeur 
		\item Accès à une valeur à partir de la clé
		\item Suppression d'une paire clé-valeur
		\item Incrémentation d'une valeur à partir de la clé
		\item Décrémentation d'une valeur à partir de la clé
	\end{itemize}
	\vspace{20px}
	Quelques autres opérations classiques sur les listes sont disponibles pour certains moteurs : ajout d'un élément à une liste, suppression d'un élément, comptage du nombre d'éléments d'une liste\dots\\

	La figure~\ref{commandesRedis} montre quelques exemples de commandes sous Redis, une base de données clé-valeur.

	\begin{listing}[H]
		\inputminted{text}{code/commandesRedis.txt}
		\caption{Quelques exemples de commandes basiques de Redis}
		\label{commandesRedis}
	\end{listing}

\subsection{Cas d'utilisation}
	Une base de données clé-valeur n'est pratiquement jamais utilisée comme seul moyen de stockage d'une application. En effet, les bases de données clé-valeur ont la plupart du temps leur stockage en RAM pour accélérer les requêtes. Le fait d'avoir les données en RAM accélère grandement les requêtes mais ne permet pas la persistance des données, contrairement à un disque dur classique. On utilise donc généralement une base de données clé-valeur en complément d'une autre base de données, relationnelle ou non. Voici quelques cas d'utilisation où une base de données clé-valeur est particulièrement adaptée :

	\begin{itemize}
	 	\item \textbf{Afficher les derniers éléments sur une page.} On peut imaginer vouloir afficher sur une page d'accueil les derniers articles de chaque catégorie d'un forum. Ceci peut se faire à l'aide d'une base de données clé-valeur en utilisant une liste où l'on stocke une partie des champs ou seulement la clé primaire des derniers articles de chaque catégorie. Dès qu'un article est publié dans une catégorie, on l'ajoute (uniquement les champs utiles, ou seulement sa clé primaire) en tête de liste. On évite ainsi un tri descendant assez fréquent sur une autre base de données.
	 	\item \textbf{Compter des éléments.} On peut imaginer avoir des commentaires sur des articles. Compter le nombre de commentaires associés à un article requiert l'utilisation d'un agrégat. On pourrait gérer le nombre de commentaires d'un article à l'aide d'une base de données clé-valeur et d'une clé propre à chaque article où l'on stocke le nombre de commentaires associés à l'article. Cette valeur est mise à jour à l'aide d'incrémentations et de décrémentations lors de l'ajout ou de la suppression d'un commentaire sur un article.
	 	\item \textbf{Cache d'une autre base de données.} On peut décider de stocker dans une base de données clé-valeur les éléments les plus souvent demandés d'une autre base de données. Ceci peut être géré facilement par un algorithme de type LRU\footnote{LRU : \textit{Least recently used}.} ou en spécifiant un temps d'expiration à chaque paire clé-valeur. On garantira ainsi un temps d'accès plus rapide aux données fréquemment demandées.
	 \end{itemize} 

\subsection{Acteurs principaux}
	La popularité des bases de données est donnée par DB-engines\cite{db_engines_key_value} pour le mois de septembre 2014.

	\begin{enumerate}
		\item \textbf{Redis}. Première version en avril 2009, écrit en C, open-source sous licence BSD. Le développement a été sponsorisé par VMware et Pivotal Software. Utilisé par The Guardian, GitHub, Stack Overflow, YouPorn, Twitter etc.\cite{Wikipedia_redis}
		\item \textbf{Memcached}. Première version en mai 2003. Écrit en C, open-source sous licence BSD. Le développement a été sponsorisé par Danga Interactive. Utilisé par Reddit, Facebook, Orange, Tumblr, Wikipedia etc. Il est connu pour utiliser une gigantesque table de hash, distribuée sur plusieurs machines. Quand la table est pleine, les données sont supprimées à l'aide de la méthode du LRU\footnote{LRU : \textit{Least recently used}.}.\cite{Wikipedia_memcached}
		\item \textbf{Riak}. Première version en août 2009, écrit en Erlang, open-source sous licence Apache 2.0. Le développement est assuré par Basho Technologies, qui propose une offre payante cloud. Utilisé par AT\&T, Boeing, Rovio, Yahoo! etc.\cite{Wikipedia_riak}
	\end{enumerate}
