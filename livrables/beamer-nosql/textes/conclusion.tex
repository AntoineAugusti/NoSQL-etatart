\section{Conclusion}

	%% Bonnes pratiques
	\subsection{Bonnes pratiques}
		\begin{frame}
			\frametitle{Les bonnes pratiques du NoSQL}

			\begin{itemize}
				\item Ne pas commencer par du NoSQL ;
				\item Bien connaître ses données ;
				\item Ne pas avoir peur de la redondance ;
				\item Ne pas trop redonder ;
				\item Et surtout, bien quantifier ses besoins.
			\end{itemize}


			\begin{block}{RTFM}
				La majorité des bases de données NoSQL connues sur le marché possèdent une très bonne documentation, souvent faite à destination de personnes venant du monde du relationnel.
			\end{block}
		\end{frame}

		\begin{frame}
			\frametitle{D'autres cas d'usage du NoSQL}
			\begin{alertblock}{Première ligne en prod', mais pas que}
				Le NoSQL ne permet pas uniquement de contenir la charge, de répondre aux problématiques du distribué.
			\end{alertblock}

			\vspace{10px}

			D'autres attraits du NoSQL:
			\begin{itemize}
				\item \textbf{Dénormalisation:} business intelligence, machine learning, data mining (BDD colonnes, documents)\dots
				\item \textbf{Stockage:} entrepôt d’agrégation de données (BDD colonnes, documents)
				\item \textbf{Modélisation:} systèmes de recommandation (BDD graphe)
				\item \dots
			\end{itemize}
		\end{frame}

	%% NoSQL ou relationnel ?
	\subsection{NoSQL ou relationnel ?}
		\begin{frame}
			\frametitle{NoSQL ou relationnel ?}

			\begin{alertblock}{Les deux mon capitaine !}
				Les bases de données NoSQL ont été inventées afin de résoudre des problèmes insolubles par les bases de données relationnelle et non \textbf{pas pour les remplacer}.
			\end{alertblock}

			\vspace{20px}

			\begin{tabular}{|l|l|}
				\hline
				\textbf{NoSQL} & \textbf{Relationnel} \\ \hline\hline
				stockage de masse & stockage fiable \\ \hline
				données diverses & données formatées \\ \hline
				scalabilité horizontale & scalabilité verticale \\ \hline
			\end{tabular}
		\end{frame}

		\begin{frame}
			\frametitle{Choix du type de BD NoSQL}

			\begin{tabular}{|p{0.20\textwidth}|p{0.40\textwidth}|p{0.30\textwidth}|}
				\hline
				& Modélisation & Cas d'utilisation \\\hline
				Clé / valeur
				& Modélisation simple, permettant d'indexer des informations diverses via une clé
				& Mise en cache  \\\hline
				Documents
				& Modélisation souple permettant de stocker des documents au format JSON dans des collections
				& Stockage de masse \\\hline
				Graphes
				& Modélisation optimisée pour les problèmes de graphes
				& Stockage provisoire pour traiter les données \\\hline
			\end{tabular}

			\vspace{10px}

			Non exhaustif: il existe d'autres types de BDD NoSQL !
		\end{frame}
