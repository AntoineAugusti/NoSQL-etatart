\subsection{La dénormalisation des données}
	Dans une relation entre deux schémas A et B, la dénormalisation consiste à dupliquer certaines données de la table B dans la table A, dans un souci d'optimisation des requêtes. On pourra alors se contenter d'interroger la table A sans avoir à faire une jointure entre A et B.\\

	Les bases de données NoSQL utilisent parfois ce principe lors de la modélisation de la base. Dans une modélisation relationnelle, ceci est à éviter le plus possible, comme recommandé par les formes normales.

\subsection{La redondance dans le monde du NoSQL}
\label{subsec:redondanceNoSQL}
	Nous avons vu ce qu'était le principe de dénormalisation. L'absence de relation entre tables induit un volume de données beaucoup plus grand car les informations sont parfois stockées de façon redondante pour être accessibles plus facilement. Il est bien sûr possible de simuler des relations entre les enregistrements.\\

	Contrairement au modèle relationnel, la redondance n'est pas une mauvaise pratique dans le monde du NoSQL. De nos jours, l'espace disque n'est pas cher et nous pouvons nous permettre de stocker des téraoctets de données sans problème. En revanche, l'accès à ces données peut se trouver ralenti car les données ne sont plus stockées sur une seule machine mais sur des systèmes de fichiers distribués. Il est donc préférable de multiplier l'information à plusieurs endroits différents. Les données sont ainsi sécurisées et n'ont aucun risque d'être perdues mais, plus important encore, elles sont accessibles beaucoup plus rapidement.\\

	D'un autre côté, la multiplication de l'information pose des problèmes lors de l'écriture. Le modèle relationnel est performant de ce point de vue car l'information est toujours présente à un seul endroit de manière cohérente. À partir du moment où la même donnée est présente à deux emplacements différents, sa mise à jour devient plus complexe et peut entraîner des incohérences.\\

	\begin{itemize}
		\item \textbf{Si la donnée est sauvegardée à plusieurs endroits physiques} : c'est le problème du système de fichier distribué, ce dernier doit minimiser les temps de mise à jour. Pour cela, la connexion entre les différents serveurs du clusters de stockage doit être très importante, on parle de connexion 10 Gbits/s au minimum. Cette vitesse permet de gérer de gros volumes de données, en limitant les temps d'incohérence de la base (lorsque la donnée a été mise à jour sur l'un des serveurs et pas sur l'autre).
		\item \textbf{Si la donnée est sauvegardée à plusieurs endroits logiques} : c'est le problème des développeurs et de l'architecte base de données. Les développeurs doivent prendre en compte les différents emplacements où est présente l'information afin de mettre à jour la donnée de manière cohérente. Il est possible d'utiliser les bases de données NoSQL à des fins différentes. Certaines applications peuvent se recouper comme par exemple une base de données utilisée pour du cache et une base de données utilisée pour du stockage. Dans une telle configuration, la modification d'une donnée dans la base de données de stockage doit être répercutée sur la base de données de cache si l'on veut assurer une cohérence. Cette modification peut être coûteuse en fonction de la modélisation choisie lors de la conception de l'architecture. Il est aisé d'oublier de répercuter ce changement partout, ce qui peut entraîner des comportements anormaux.
	\end{itemize}