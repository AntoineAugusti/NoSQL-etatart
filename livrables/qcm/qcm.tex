\documentclass[twocolumn,landscape,8pt,a4paper]{article}

\usepackage{geometry}
\usepackage{multicol}
\usepackage[utf8x]{inputenc}
\usepackage[T1]{fontenc}
\usepackage[francais,bloc]{automultiplechoice}
\usepackage{datatool}
\usepackage{amssymb}
\usepackage{graphicx}
\usepackage[francais]{babel}
\usepackage{listings}


\geometry{hmargin=1cm,headheight=2cm,headsep=.3cm,footskip=0.5cm,top=2.5cm,bottom=1cm}

%\baremeDefautM{b=0,m=-2,d=1,p=-1}
\baremeDefautM{formula=(NMC>0 ? -1 : NBC/NB)}
\baremeDefautS{e=-1,v=0,b=1,m=-1}
\AMCinterIrep=-1mm

\scriptsize
\AMCboxDimensions{size=2mm,down=.2ex}

\DTLsetseparator{;}

\newcommand{\letitre}{}
\newcommand{\ladate}{//}
\DTLloaddb{listeetudiants}{listeASI4.csv}


%%%%%%%%%%%%%%%%%%%%%%%%%%%%%%%%%%%%%%%%%%%%%%%%%%%%%%%%%%%%%%%%%%%%%%
% le fichier doit contenir au moins trois colonnes sous la forme :
% première ligne obligatoire : nom;prenom;numero
% les champs sont séparés par des ";"
% Les numéros doivent tous avoir le même nombre de chiffres (ex: 01)
%%%%%%%%%%%%%%%%%%%%%%%%%%%%%%%%%%%%%%%%%%%%%%%%%%%%%%%%%%%%%%%%%%%%%%

\begin{document}

%%%%%%%%%%%%%%%%%%%%%%%%%%%%%%%%%%%%%%%%%%%%%%%%%%%%%%%%%%%%%%%%%%%%%%
% \element défini le groupe de la question
% ne pas oublier le MELANGE et la RESTITUION (cf. fin du doc)
% \begin{question} et \begin{questionmult} les idq doivent être
% uniques
%%%%%%%%%%%%%%%%%%%%%%%%%%%%%%%%%%%%%%%%%%%%%%%%%%%%%%%%%%%%%%%%%%%%%%



\element{idg}{
\begin{questionmult}{q0}
  Quels les principaux inconvénients d'une architecture classique utilisant une BD ?
  \begin{reponses}
    \mauvaise{Elle est rapide à mettre en place}
    \bonne{Il faut être un expert en BD}
    \bonne{L'optimisation horizontale est difficile}
    \mauvaise{Elle coûte cher à déployer}
    \end{reponses}
\end{questionmult}
}

\element{idg}{
\begin{question}{q1}
  La stack LAMP est composée de\dots
  \begin{reponses}
    \bonne{Linux, Apache, MySQL, PHP}
    \mauvaise{Linux, Apache, MongoDB, PHP}
    \mauvaise{Linux, AngularJS, MySQL, PHP}
    \mauvaise{Léonard, Amélie, Morgane, Patrick}
    \mauvaise{J'ai lu le mot steak. Je relis attentivement la question.}
  \end{reponses}
\end{question}
}


\element{idg}{
\begin{question}{q3}
  Qu'est-ce qu'un ORM ?
  \begin{reponses}
    \bonne{Un outil permettant de simuler une BD objet à partir d'une BD relationnelle}
    \mauvaise{Un outil permettant d'optimiser les requêtes facilement}
    \mauvaise{Un outil pour ne pas écrire de requêtes SQL}
    \mauvaise{Un Outil Remarquablement Minutieux}
  \end{reponses}
\end{question}
}

\element{idg}{
\begin{questionmult}{q4}
  Pourquoi avoir eu besoin d'autre chose que des BD relationnelles ?
  \begin{reponses}
    \bonne{Pour pouvoir gérer un grand volume de données}
    \bonne{Pour pouvoir supporter un grand nombre de transactions}
    \mauvaise{Parce que les bases de données relationnelles sont mauvaises}
    \mauvaise{Parce que le SQL c'est trop compliqué à connaître}
  \end{reponses}
\end{questionmult}
}

\element{idg}{
\begin{questionmult}{q5}
  Quelles affirmations sont vraies pour une BD NoSQL ?
  \begin{reponses}
    \bonne{Il n'est pas nécessaire de respecter les formes normales}
    \mauvaise{Plus besoin d'optimiser les requêtes, tout est automatique}
    \bonne{On doit choisir son type de BD NoSQL en fonction du besoin}
    \mauvaise{Il est impossible d'utiliser du relationnel et du NoSQL dans une même application}
  \end{reponses}
\end{questionmult}
}

\element{idg}{
 \begin{question}{q6}
   L'optimisation verticale consiste à :
   \begin{reponses}
     \mauvaise{Empiler des serveurs les uns sur les autres}
     \bonne{Améliorer les performances physiques de son serveur}
     \mauvaise{Ajouter d'autres serveurs}
     \mauvaise{Tuer sur les processus inutiles sur son serveur}
   \end{reponses}
 \end{question}
}

\element{idg}{
\begin{question}{q7}
  L'optimisation horizontale consiste à :
  \begin{reponseshoriz}
     \mauvaise{Empiler des serveurs les uns sur les autres}
     \mauvaise{Améliorer les performances physiques de son serveur}
     \bonne{Ajouter d'autres serveurs}
     \mauvaise{Tuer sur les processus inutiles sur son serveur}  \end{reponseshoriz}
\end{question}
}

\element{idg}{
 \begin{questionmult}{q8}

   \begin{reponseshoriz}
     \bonne{}
     \bonne{}
     \mauvaise{}
   \end{reponseshoriz}
 \end{questionmult}
}

\element{idg}{
 \begin{questionmult}{q9}

   \begin{reponses}
     \bonne{}
     \bonne{}
     \mauvaise{}
   \end{reponses}
 \end{questionmult}
 }
 \element{idg}{
  \begin{questionmult}{q10}

    \begin{reponses}
     \mauvaise{}
     \mauvaise{}
     \bonne{}
     \end{reponses}
 \end{questionmult}
 }
 \element{idg}{
   \begin{question}{q11}

   \begin{reponseshoriz}
     \mauvaise{}
     \bonne{}
     \mauvaise{}
   \end{reponseshoriz}
 \end{question}
 }

 \element{idg}{
\begin{questionmult}{q13}

   \begin{reponseshoriz}
     \bonne{}
     \bonne{}
     \mauvaise{}
   \end{reponseshoriz}
 \end{questionmult}
 }
 \element{idg}{
\begin{question}{q14}

     \begin{reponses}
       \bonne{}
       \mauvaise{}
       \mauvaise{}
     \end{reponses}
   \end{question}
}
\element{idg}{
 \begin{question}{q15}

   \begin{reponseshoriz}
     \bonne{}
     \mauvaise{}
     \mauvaise{}
   \end{reponseshoriz}
 \end{question}
}
\element{idg}{
    \begin{question}{q16}

     \begin{reponses}
       \bonne{}
       \mauvaise{}
       \mauvaise{}
     \end{reponses}
  \end{question}
 }







\exemplaire{20}{

  \begin{center}\large
    \textbf{\letitre}


    \ladate
  \end{center}

%\def\multiSymbole{$\clubsuit$}
\def\multiSymbole{}

{
\AMCcodeHspace=.3em
\AMCcodeVspace=.2em
\AMCcodeBoxSep=.1em
\begin{minipage}{0.4\linewidth}
Codez votre numéro (cf. liste) :

\AMCcodeH{etu}{2}
\end{minipage}
\champnom{\fbox{
\begin{minipage}{.4\linewidth}
Nom et prénom :

\vspace*{.5cm}\dotfill
\vspace*{.5cm}\dotfill
\vspace*{1mm}
\end{minipage}
}}
}



\begin{minipage}[c]{0.1\linewidth}
  \includegraphics[width=1cm]{danger.pdf}
\end{minipage}
\begin{minipage}[c]{0.7\linewidth}
  \begin{itemize}\em
  \item les cases doivent être noircies $\blacksquare$ \textbf{au stylo noir}
  \item ne pas griffonner ou corriger le questionnaire
  \end{itemize}
\end{minipage}\hfill



\vspace{1ex}

%%%%%%%%%%%%%%%%%%%%%%%%%%%%%%%%%%%%%%%%%%%%%%%%%%%%%%%%%%%%%%%%%%%%%%
% MELANGE DES GROUPES ET RESTITUTION

\melangegroupe{idg}

\restituegroupe{idg}

%%%%%%%%%%%%%%%%%%%%%%%%%%%%%%%%%%%%%%%%%%%%%%%%%%%%%%%%%%%%%%%%%%%%%%






\clearpage
%%%%%%%%%%%%%%%%%%%%%%%%%%%%%%%%%%%%%%%%%%%%%%%%%%%%%%%%%%%%%%%%%%%%%%
% LISTE DES ETUDIANTS AVEC LEUR CODE
%%%%%%%%%%%%%%%%%%%%%%%%%%%%%%%%%%%%%%%%%%%%%%%%%%%%%%%%%%%%%%%%%%%%%%

% \normalsize
% \begin{tabular}{llc|llc}
% \textbf{Nom} & \textbf{Prénom} & \textbf{Numéro} &\textbf{Nom} & \textbf{Prénom} & \textbf{Numéro}%

% \DTLforeach*{listeetudiants}{\nom=nom,\prenom=prenom,\numero=numero}{%
% \DTLifoddrow{\\}{&}%
% \nom & \prenom & \numero}%
% \end{tabular}
% \newpage

% \vspace*{\fill}
% \begin{center}\large
% \textbf{Exemples de codage}

%   \begin{minipage}{0.4\linewidth}
%     \textbf{Étudiant 02 :}

%     \noindent$\blacksquare$0  $\square$1  $\square$2  $\square$3  $\square$4  $\square$5  $\square$6  $\square$7  $\square$8  $\square$9\\
%     $\square$0  $\square$1  $\blacksquare$2  $\square$3  $\square$4  $\square$5  $\square$6  $\square$7  $\square$8  $\square$9\\

%     \textbf{Étudiant 12 :}

%     \noindent$\square$0  $\blacksquare$1  $\square$2  $\square$3  $\square$4  $\square$5  $\square$6  $\square$7  $\square$8  $\square$9\\
%     $\square$0  $\square$1  $\blacksquare$2  $\square$3  $\square$4  $\square$5  $\square$6  $\square$7  $\square$8  $\square$9\\
%   \end{minipage}
% \end{center}
% \vfill

% }
\end{document}