\documentclass[a4paper,10pt]{article}
\usepackage{fullpage}
\usepackage{vmargin}
\usepackage[utf8]{inputenc}
\usepackage[french]{babel}
\usepackage{amsmath}
\usepackage{pifont}
\usepackage{verbatim}
\usepackage{graphicx}
\usepackage{tikz}
\usepackage{csquotes}
\usepackage{listings}
\lstset{literate=
  {á}{{\'a}}1 {é}{{\'e}}1 {í}{{\'i}}1 {ó}{{\'o}}1 {ú}{{\'u}}1
  {Á}{{\'A}}1 {É}{{\'E}}1 {Í}{{\'I}}1 {Ó}{{\'O}}1 {Ú}{{\'U}}1
  {à}{{\`a}}1 {è}{{\`e}}1 {ì}{{\`i}}1 {ò}{{\`o}}1 {ù}{{\`u}}1
  {À}{{\`A}}1 {È}{{\'E}}1 {Ì}{{\`I}}1 {Ò}{{\`O}}1 {Ù}{{\`U}}1
  {ä}{{\"a}}1 {ë}{{\"e}}1 {ï}{{\"i}}1 {ö}{{\"o}}1 {ü}{{\"u}}1
  {Ä}{{\"A}}1 {Ë}{{\"E}}1 {Ï}{{\"I}}1 {Ö}{{\"O}}1 {Ü}{{\"U}}1
  {â}{{\^a}}1 {ê}{{\^e}}1 {î}{{\^i}}1 {ô}{{\^o}}1 {û}{{\^u}}1
  {Â}{{\^A}}1 {Ê}{{\^E}}1 {Î}{{\^I}}1 {Ô}{{\^O}}1 {Û}{{\^U}}1
  {œ}{{\oe}}1 {Œ}{{\OE}}1 {æ}{{\ae}}1 {Æ}{{\AE}}1 {ß}{{\ss}}1
  {ç}{{\c c}}1 {Ç}{{\c C}}1 {ø}{{\o}}1 {å}{{\r a}}1 {Å}{{\r A}}1
  {€}{{\EUR}}1 {£}{{\pounds}}1
}
\DeclareUnicodeCharacter{00A0}{~}
\usetikzlibrary{positioning,shadows,arrows,fit,automata,fadings,patterns,shapes}

% Couleurs prédéfinies
\colorlet{vertP}{green!10!white}
\colorlet{vertF}{green!90!black}
\colorlet{vertM}{green!40!white}
\definecolor{gazon}{RGB}{58,157,35}
\colorlet{bleuP}{blue!10!white}
\colorlet{bleuF}{blue!90!black}
\colorlet{bleuM}{blue!40!white}
\colorlet{rougeP}{red!10!white}
\colorlet{rougeM}{red!40!white}
\colorlet{rouge}{red!80!black}
\colorlet{rougeF}{red!90!black}
\colorlet{bleu}{blue!80!black}
\colorlet{vert}{green!80!black}
\colorlet{jaune}{yellow!80!black}
\colorlet{jauneF}{yellow!90!black}
\colorlet{jauneM}{yellow!40!white}
\colorlet{jauneP}{yellow!10!white}
\colorlet{grisP}{gray!20!white}
\definecolor{grisclair}{RGB}{206,206,206}
\definecolor{orangebis}{RGB}{222, 41, 22}
\definecolor{roux}{RGB}{173,79,9}
\definecolor{lilas}{RGB}{182,102,210}
\colorlet{lilasP}{lilas!20!white}
\colorlet{lilasM}{lilas!40!white}
\colorlet{jauneJoli}{yellow!50!white}
\definecolor{bordeaux}{RGB}{109,7,26}
\definecolor{magenta}{RGB}{255,0,255}

\usepackage[tikz]{bclogo}
\renewcommand\styleSousTitre[1]{\scriptsize\textsc{#1}}
\usepackage{placeins}
\usepackage{url}

\title{Correction TP bases de données NoSQL}
\date{}
\author{Antoine Augusti \& Thibaud Dauce}

\begin{document}

\maketitle
\sloppy

\section{Redis}
  \subsection{Découverte de Redis}
    Pas de correction nécessaire.

  \subsection{Un mini réseau social}

    \begin{enumerate}
      \item Pour modéliser un utilisateur nous allons utiliser un dictionnaire (\textit{hash map}). Nous allons utiliser une clé par utilisateur, et pour identifier un utilisateur, nous mettrons son identifiant dans le nom de la clé. Par exemple, les données de l'utilisateur \texttt{42} seront dans la clé \texttt{user:42}.
      \item \texttt{HMSET user:42 pseudonyme bob pays FR date\_naissance 1980-11-22}
      \item
        \begin{enumerate}
          \item Nous allons utiliser un compteur car c'est une structure de donné atomique qui permet de modéliser idéalement des identifiants qui sont incrémentés de un à chaque inscription. On pourrait nommer cette clé \texttt{next\_user\_id} par exemple.
          \item Avant le lancement de notre réseau social, il faudra créer une clé contenant ce compteur et l'initialiser à 0. Ainsi, lors de l'inscription du premier utilisateur, celui-ci aura l'identifiant 1.
          \item On devrait incrémenter le compteur pour obtenir un identifiant \texttt{INCR next\_user\_id} et utiliser la valeur de retour de la commande comme identifiant de l'utilisateur à créer dans un dictionnaire (commande \texttt{HMSET} vue précédemment).
        \end{enumerate}
      \item Dans notre réseau social, les utilisateurs peuvent suivent d'autres utilisateurs et peuvent être suivis par d'autres utilisateurs. Cette relation est unidirectionnelle : si je suis un autre utilisateur, je ne suis pas automatiquement suivi par cet autre utilisateur en retour.
        \begin{enumerate}
          \item Nous pouvons utiliser une liste, un ensemble ou un ensemble ordonné. Nous aurons 2 clés par utilisateur : une clé contenant les personnes suivies (\texttt{followings:42}) et une clé contenant les personnes qui suivent un utilisateur en question (\texttt{followers:42}).
          \item Les ensembles sont idéaux car il ne devrait pas être possible pour un utilisateur de suivre plusieurs fois un autre. On cherche donc à garder l'unicité, ce qui est l'objectif d'un ensemble.
          \item  On peut utiliser un ensemble ordonné pour conserver l'ordre des actions de suivi des utilisateurs.
          \item \texttt{ZADD followers:50 1401267618 42} et \texttt{ZADD followings:42 1401267618 50}
        \end{enumerate}
      \item On aurait une table \texttt{followers} avec au moins 4 colonnes : \texttt{id}, \texttt{user\_id}, \texttt{follower\_id}, \texttt{created\_at}. \texttt{id} est clé primaire, \texttt{user\_id} et \texttt{follower\_id} sont clés étrangères de la colonne ID de \texttt{users} et ont une contrainte d'unicité.
    \end{enumerate}

\section{JSON avec PostgreSQL}

  \subsection{Création de la table}
    \begin{enumerate}
      \item
      \begin{lstlisting}[language=SQL, basicstyle=\ttfamily, gobble=6, tabsize=2]
      create table products (title varchar, price integer, data jsonb);
      \end{lstlisting}
      \item Insérez les écrans suivants :
          \begin{itemize}
            \item[$\bullet$]
            \begin{lstlisting}[language=SQL, basicstyle=\ttfamily, gobble=12, tabsize=2, showstringspaces=false]
            INSERT INTO products VALUES ('Iiyama', 14999, '{
                "diagonal": 24,
                "technology": "LED",
                "connectivity":{
                    "VGA":1
                },
                "gsync": false
            }');
            \end{lstlisting}
            \item[$\bullet$]
            \begin{lstlisting}[language=SQL, basicstyle=\ttfamily, gobble=12, tabsize=2, showstringspaces=false]
            INSERT INTO products VALUES ('Acer', 45999, '{
                "diagonal": 27,
                "technology": "LED",
                "connectivity":{
                    "HDMI":1
                },
                "gsync": true
            }');
            \end{lstlisting}
            \item[$\bullet$]
            \begin{lstlisting}[language=SQL, basicstyle=\ttfamily, gobble=12, tabsize=2, showstringspaces=false]
            INSERT INTO products VALUES ('HP', 14404, '{
                "diagonal": 24,
                "technology": "LED",
                "connectivity":{
                    "HDMI":1,
                    "VGA":1,
                    "DVI":1
                },
                "gsync": false
            }');
            \end{lstlisting}
            \item[$\bullet$]
            \begin{lstlisting}[language=SQL, basicstyle=\ttfamily, gobble=12, tabsize=2, showstringspaces=false]
            INSERT INTO products VALUES ('Dell', 2400, '{
                "diagonal": 17,
                "technology": "LCD"
            }');
            \end{lstlisting}
         \end{itemize}
    \end{enumerate}

  \subsection{Requêtes sur le JSON}
    \begin{lstlisting}[language=SQL, basicstyle=\ttfamily, gobble=4, tabsize=2, showstringspaces=false]
    SELECT title FROM products WHERE (data->>'diagonal')::int = 24;
    SELECT title FROM products WHERE (data->>'diagonal')::int >= 24;
    SELECT title FROM products WHERE (data->'connectivity'->>'HDMI')::int >= 1;
    UPDATE products SET data = jsonb_set(data, '{connectivity, DVI}', '2')
    WHERE title = 'HP';
    UPDATE products SET data = data - 'gsync' WHERE title = 'Iiyama';
    \end{lstlisting}

  \subsection{Méta-requêtes sur le JSON}
    \begin{lstlisting}[language=SQL, basicstyle=\ttfamily, gobble=4, tabsize=2, showstringspaces=false]
    -- Toutes les cases
    SELECT DISTINCT
      key
    FROM (
      SELECT (jsonb_each(data)).*
      FROM products
    ) as key_values
    WHERE jsonb_typeof(value) = 'boolean';

    -- Tous les sliders
    SELECT DISTINCT
      key,
      min(value::text::float),
      max(value::text::float)
    FROM (
      SELECT (jsonb_each(data)).*
      FROM products
    ) as key_values
    WHERE jsonb_typeof(value) = 'number'
    GROUP BY key;

    -- Choix multiples
    SELECT
      key,
      array_agg(value::text)
      FROM (
        SELECT DISTINCT
          (jsonb_each(data)).*
        FROM products
      ) as key_values
      WHERE jsonb_typeof(value) = 'string'
    GROUP BY key;
    \end{lstlisting}

  \subsection{Recherche}
    \begin{lstlisting}[language=SQL, basicstyle=\ttfamily, gobble=4, tabsize=2, showstringspaces=false]
    -- Ajouter la colonne
    ALTER TABLE products ADD COLUMN description text;

    -- Mise à jour des produits existants
    UPDATE products SET description = 'Super écran' WHERE title = 'Iiyama';
    UPDATE products SET description = 'Bon pour les joueurs' WHERE title = 'Acer';
    UPDATE products SET description = 'Une bonne dalle presque comme le Iiyama'
      WHERE title = 'HP';
    UPDATE products SET description = 'Un vieil écran' WHERE title = 'Dell';

    -- Expérimentations avec tsvector et la langue
    SELECT to_tsvector(description) FROM products;
    SELECT to_tsvector('english', description) FROM products;

    -- Recherche du terme écran avec et sans tsvector
    SELECT title FROM products WHERE to_tsvector(description) @@ 'écran';
    SELECT title FROM products
      WHERE to_tsvector(description) @@ to_tsquery('écran');

    -- Recherche du terme bonne avec et sans tsvector
    SELECT title FROM products WHERE to_tsvector(description) @@ 'bonne';
    SELECT title FROM products
      WHERE to_tsvector(description) @@ to_tsquery('bonne');

    -- Recherche avec un poids plus important pour le titre
    -- que pour la description
    SELECT title FROM products
    ORDER BY ts_rank(
        setweight(to_tsvector(title), 'A') ||
        setweight(to_tsvector(description), 'B'),
        to_tsquery('Iiyama')
    ) DESC;

    -- Recherche en excluant les résultats non pertinant
    SELECT title FROM products
    WHERE to_tsvector(title) || to_tsvector(description) @@ to_tsquery('Iiyama')
    ORDER BY ts_rank(
        setweight(to_tsvector(title), 'A') ||
        setweight(to_tsvector(description), 'B'),
        to_tsquery('Iiyama')
    ) DESC;
    \end{lstlisting}
\end{document}
