\subsection{Le théorème CAP}
\label{subsec:theoremeCAP}
	Classiquement, la focalisation de l'informatique distribuée était faite sur le calcul, et non sur les données. C'est pour des besoins de calcul qu'on créait des clusters, pour atteindre une puissance de calcul multipliée par le nombre de nœuds. Eric Brewer, professeur de l'université de Californie affirme que le calcul distribué est relativement facile. Le paradigme MapReduce abordé dans la partie~\ref{subsec:principeMapReduce} est maintenant éprouvé et n'est plus particulièrement difficilement à mettre en place. Ce qui est difficile, c'est la distribution des données. Le professeur Brewer énonce le théorème CAP, une abréviation de trois propriétés :
	\vspace{10px}
	\begin{itemize}
		\item \textbf{\textit{Consistency}} (cohérence) : tous les nœuds sont à jour sur les données au même moment ; 
		\item \textbf{\textit{Availability}} (disponibilité) : la perte d'un nœud n'empêche pas le système de fonctionner et de servir l'intégralité des données ; 
		\item \textbf{\textit{Partition tolerance}} (résistance au morcellement) : chaque nœud doit pouvoir fonctionner de manière autonome. 
	\end{itemize}
	\vspace{20px}
	L'expression \textit{Partition tolerance} n'est pas très claire. On entend par là qu'un système partitionné doit pouvoir survivre aux difficultés du réseau. D'après Seth Gilbert\footnote{Seth Gilbert est un doctorant à l'École polytechnique fédérale de Lausanne (EPFL).} et Nancy Lynch\footnote{Nancy Lynch est une professeure au Massachusetts Institute of Technology (MIT).}, \enquote{le système doit être capable de survivre aux aléas de la perte de partitions ou de communication entre les partitions. Il ne s'agit pas d'une garantie totale d'accès à partir d'un nœud, mais bien d'une capacité à la survie et à la résilience aussi poussée que possible.}\\

	Les bases de données distribuées ne peuvent satisfaire que deux de ces critères au plus.

\subsection{Les propriétés BASE}
	Nous avons pu voir grâce au théorème CAP dans la partie \ref{subsec:theoremeCAP} que les transactions sur des systèmes distribuées posaient de nouveaux challenges. Dans des systèmes distribuées, les propriétés ACID (présentées dans la partie \ref{subsec:proprietesACID}) apparaissent comme impossibles à satisfaire entièrement vu que notre préoccupation est la disponibilité et une haute performance, dont le but est de supporter un fort volume de transactions.\\

	Par exemple, supposons que l'on vende des livres en ligne et que l'on affiche fièrement combien de livres nous possédons en stock. Chaque fois qu'un visiteur souhaite acheter un livre, nous mettons un verrou sur une partie des données jusqu'à ce qu'il paye, de manière à ce que tous les autres visiteurs voient le bon nombre de livres en stock. Ceci fonctionne bien si vous êtes un libraire d'une ville de taille moyenne, mais devient impossible pour Amazon par exemple. On revient alors aux problématiques du théorème CAP qui indique qu'on est obligé de sacrifier une des trois propriétés pour assurer les deux autres.\\

	En connaissance de ceci, Eric Brewer propose une alternative aux propriétés ACID, les propriétés BASE, qui sont énoncées de la manière suivante :
	\vspace{10px}
	\begin{itemize}
	 	\item \textbf{\textit{Basic Availability.}} Le système distribué est disponible dans son ensemble, bien qu'il soit inévitable que certains nœuds ne soient pas disponibles par moment.
	 	\item \textbf{\textit{Soft state.}} L'état du système distribué peut changer au cours du temps, même sans nouvelles transactions. Ceci s'explique par la propriété précédente : certaines transactions peuvent être propagées tardivement, résultant en une mise à jour différée des données sur certaines parties du système.
	 	\item \textbf{\textit{Eventual consistency.}} Indique que le système sera cohérent au bout d'un temps, sous réserve que celui-ci ne reçoive plus de nouvelles transactions.
	 \end{itemize}
	 \vspace{20px}
	 On remarque que les propriétés BASE assurent bien moins de cohérence des données et des transactions que les propriétés ACID. Pour s'assurer que le système distribuée évolue vers un état stable il devient alors nécessaire de résoudre des conflits provenant des différents états des nœuds du système. C'est seulement à ce prix qu'il est actuellement possible de concevoir une importante architecture de stockage de données distribuée. 