\subsection{Modélisation}
	Une \bddGraphe{} enregistre des données dans un graphe. Pour simplifier, un graphe connecte des nœuds (des \enquote{entités}) entre eux à l'aide de liens, qui symbolisent les relations (les \enquote{associations}). On peut stocker les informations que l'on souhaite dans les nœuds ou dans les liens. Pour avoir une définition plus complète, il est possible de \enquote{lire} le graphe de la figure \ref{bddGrapheSVG} en suivant les flèches. On forme des phrases en partant d'un nœud puis en empruntant une arête jusqu'au prochain nœud ; par exemple \enquote{\textit{A graph database manages a graph}}.

	\begin{figure}[H]
		\centering
		\includegraphics[width=0.9\textwidth]{images/bddGraphe.png}
		\caption{Qu'est-ce qu'une \bddGraphe{} ? \cite{bddGrapheSchema}}
		\label{bddGrapheSVG}
	\end{figure}

	\subsubsection{Exemple de requête sous Neo4j}
		Neo4j est une \bddGraphe{} qui est actuellement la plus populaire de sa catégorie. Voici un exemple de graphe qui représente une partie des relations du film \enquote{\textit{The Matrix}} et \enquote{\textit{Cloud Atlas}} sur la figure \ref{grapheNeo4j}. On y retrouve les différents acteurs et les réalisateurs des films. Les nœuds sont les personnes et les liens sont les relations qu'ils ont par rapport au film. Dans notre cas, il n'existe que deux types de liens :
		\vspace{5px}
		\begin{itemize}
			\item \textit{DIRECTED} : la personne est un des réalisateurs du film pointé par la liaison ;
			\item \textit{ACTED\_IN} : la personne est un des acteurs du film pointé par la liaison.
		\end{itemize}

		\begin{figure}[H]
			\centering
			\includegraphics[width=0.9\textwidth]{images/graphe.png}
			\caption{Un graphe représentant une partie des relations pour les films \enquote{\textit{The Matrix}} et \enquote{\textit{Cloud Atlas}}.\cite{grapheNeo4j}}
			\label{grapheNeo4j}
		\end{figure}

		Imaginons que l'on veuille, à partir de ce graphe, trouver tous les acteurs jouant dans le film \enquote{\textit{The Matrix}} et le rôle qu'ils jouent. Pour réaliser ceci, nous allons utiliser Cypher, le langage de requête de Neo4j. Cypher est un langage plutôt simple, déclaratif, dont le but est d'exprimer ce qu'il faut récupérer, et non comment le récupérer. La requête Cypher à écrire est présentée dans la figure~\ref{requeteNeo4j}.

		\begin{figure}[H]
			\centering
			\includegraphics[width=0.9\textwidth]{images/requeteNeo4j.png}
			\caption{Trouver tous les acteurs du film \enquote{\textit{The Matrix}}.\cite{grapheNeo4j}}
			\label{requeteNeo4j}
		\end{figure}

		Cypher utilise l'art ASCII\footnote{L'art ASCII consiste à réaliser des images uniquement à l'aide des lettres et caractères spéciaux contenus dans le code ASCII.\cite{Wikipedia_artASCII}} pour représenter les liaisons entre des nœuds. Ainsi, la partie de requête \texttt{(A:Person)-[B:ACTED\_IN]->(C:Movie)} cherche à trouver dans le graphe un nœud A, de type \enquote{Person}, lié par la relation \enquote{ACTED\_IN} à un nœud C, de type \enquote{Movie}. Les \textit{labels} représentent les classes des nœuds ou des liaisons, les \textit{identifiers} sont les variables qui sont créées dynamiquement et les \textit{properties} sont les attributs des nœuds ou des liaisons. Les nœuds et les liaisons ont une importance égale : ils peuvent avoir des classes, des attributs et on peut appliquer des critères de recherche ainsi que des agrégats sur chacun. Une manière élégante de sélectionner un noeud ou une liaison selon un critère est présentée dans la figure \ref{raccourciSelection}.\\

		\begin{figure}[H]
			\centering
			\includegraphics[width=0.8\textwidth]{images/shortSelect.png}
			\caption{La sélection sur un attribut peut se faire plus facilement.}
			\label{raccourciSelection}
		\end{figure}

		Imaginons que l'on veuille cette fois-ci trouver les noms des personnes qui relient deux personnes qui ne sont pas amies. On a donc un graphe plutôt simple : les nœuds sont des personnes, et il n'y a qu'un seul type de liaison que l'on appelle \textit{KNOWS} quand deux personnes se connaissent. Neo4j inclut par défaut la plupart des algorithmes les plus utilisés sur les graphes, en particulier celui du plus court chemin sous un ensemble de contraintes. Neo4j supporte également les expressions rationnelles. Ainsi, pour trouver les noms des personnes qui relient deux personnes qui ne sont pas amies, on devrait exécuter la requête présentée dans la figure~\ref{connectingFriends}.

		\begin{figure}[H]
			\centering
			\includegraphics[width=0.8\textwidth]{images/connectingFriends.png}
			\caption{Recherche des noms des personnes reliant Bertrand et Hélène.}
			\label{connectingFriends}
		\end{figure}

		La dernière ligne de la requête présentée dans la figure~\ref{connectingFriends} est la plus compliquée. L'instruction \texttt{nodes(p)} permet de récupérer les nœuds du chemin le plus court trouvé. On itère ensuite sur chaque nœud de cette liste : \texttt{[n in nodes(p)[1..-1]]}. On part du nœud d'indice 1, car le nœud d'indice 0 correspond à \enquote{Bertrand} dans notre exemple. On va \enquote{jusqu'à l'index -1} car on ne connaît pas la longueur de la liste. Enfin, la dernière instruction \texttt{| n.name} indique que l'on va retourner l'attribut correspondant au nom de la personne, pour chaque nœud.

\subsection{Cas d'utilisation}
	Une \bddGraphe{} n'a pas pour vocation à être utilisée comme seul moteur de stockage, sauf dans le cas d'applications très spécifiques ou de modules d'un système d'information. Les bases de données orientée graphe sont adaptées lorsque les données manipulées possèdent un grand nombre de liaisons entre elles. Pour rappel, les bases de données orientée graphe accordent une importance égale pour les nœuds et les relations entre ces nœuds. Voici quelques cas d'utilisation où une \bddGraphe{} est particulièrement adaptée\footnote{Ces exemples sont tirés du site officiel de Neo4j : \url{http://neo4j.com}} :
	\vspace{10px}
	\begin{itemize}
	 	\item \textbf{Modélisation de réseaux.} Les interconnexions entre des équipements, virtuels ou physiques, peuvent être représentées parfaitement par un graphe. On peut imaginer représenter un réseau de distribution d'eau ou un réseau mobile par exemple. Une fois le système modélisé on peut effectuer des analyses d'impact de panne, détecter la source d'un incident, améliorer la qualité de service, prévoir des maintenances\dots
	 	\item \textbf{Réseaux sociaux.} La famille, les amis, les followers sont parfaitement représentés dans un graphe social qui permet de mettre en évidence les comportements, les influences et les groupes implicites. On peut alors réaliser des recommandations d'amis, des suggestions de partage et de l'analyse d'influence.
	 	\item \textbf{Gestion d'accès.} Qui vous êtes, à qui vous appartenez et ce que vous êtes autorisé à faire dépend des relations entre vous, une entreprise et un système.
	 	\item \textbf{Organisation hiérarchique.} Qui dirige qui ? Qui est responsable de qui ? Qui finance qui ? Toutes ces questions reviennent souvent pour une entreprise ayant une taille conséquence. Ces relations peuvent être modélisées dans une \bddGraphe{}.
	 \end{itemize}

\subsection{Acteurs principaux}
	La popularité des bases de données est donnée par DB-engines\cite{db_engines_key_value} pour le mois de novembre 2014.

	\begin{enumerate}
		\item \textbf{Neo4j}. Première version en 2007, écrit en Java, open-source sous licence GPLv3. Respecte les transactions ACID (présentées dans la partie \ref{subsec:proprietesACID}). Certains modules (comme la haute disponibilité) sont payants et sous licence AGPLv3. Le développement est assuré par Neo Technology. Utilisé par SFR, Ebay, HP, National Geographic, Walmart\dots
		\item \textbf{Titan.} Première version en 2012, écrit en Java, open-source sous licence Apache 2. Le développement est assuré par Aurelius, une équipe d'ingénieurs qui aide au déploiement de Titan. Peut respecter les propriétés ACID (présentées dans la partie \ref{subsec:proprietesACID}).
		\item \textbf{OrientDB.} Première version en 2010, écrit en Java, open-source sous licence Apache 2. C'est une base de données orientée documents mais les relations sont stockées dans un graphe. Le système peut être interrogé en SQL.
	\end{enumerate}
